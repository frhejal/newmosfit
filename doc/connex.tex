Le module \lstinline{connex} definit une relation liant un paramètre de l'un des sous-spectres avec un autre paramètre, possiblement d'un autre spectre. 
La selection se fait en choisissant une valeur non nulle pour \lstinline{IO(5)}$\neq0$.
La valeur donnée à \lstinline{IO(5)} détermine la relation choisie.
Elle doit correspondre  à  une des valeurs de l'embranchement \lstinline{select case(IO(5))} définie dans le code source de \lstinline{connex}.
Lorqu'un paramètre est liée à un autre, le parametre d'ajustement de ce paramètre (valeur de \lstinline{NB}) doit être égal à $3$.
Par exemple, avec le code suivant dans le module \lstinline{connex} :
\begin{lstlisting}
select case(io(5))
  case(0) 
    print *, 'Un appel inutile a CONNEX a ete fait'
  case(1)                       !  J.P  OXYDATION DU VERT        
    bt(3,2)=0.63_dp*bt(2,1)*bt(3,1)/bt(2,2)
  case(2)                       !  FERRITES BEATRICE                                                
    do i=2,20
      bt(1,i)=bt(1,1)+0.11_dp
    enddo 
  case(3)                       !  GAETAN
    bt(3,3)=bt(2,1)*bt(3,1)/3.0_dp/bt(2,3)
    bt(3,4)=bt(2,6)*bt(3,6)/3.0_dp/bt(2,4)
  case default
    stop 'OPTION IO(5) : VALEUR INCONNUE DANS CONNEX'
end select
\end{lstlisting}

Les connexions réalisées seront :

  \begin{itemize}
    \item si \lstinline{IO(5)}=1,  surface totale du 2\up{ème} spectre = 0.63* surface du 1\up{er} 
          (on doit déclarer pour le 2\up{ème} spectre : \lstinline{NB(3)}$=3$, 
    \item si \lstinline{IO(5)}=2,  \lstinline{DI} du \lstinline{I}ème spectre =  \lstinline{DI} du 1\up{er}+0.11 (pour $2\le $\lstinline{I}$\le 20$)
     (on doit déclarer pour chacun de ces spectres : \lstinline{NB(1)}$=3$), 
    \item si \lstinline{IO(5)}=3, surface du 3\up{ème} spectre = $\frac{1}{3}$ surface du 1\up{er},
    surface du 4\up{ème} spectre = $\frac{1}{3}$ surface du 6\up{ème}
    (on doit déclarer pour les spectres 3 et 4 : \lstinline{NB(3)}$=3$)
  \end{itemize}


Pour ajouter une relation personnalisée de numéro N, l'utilisateur doit modifier le code source pour y ajouter le cas \lstinline{case(N)} et recompiler le code. 

