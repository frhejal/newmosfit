\newpage
\subsection{Calcul des spectres théoriques}
{\it Cette section reprend en grande partie, parfois littéralement, le chapitre 4 de la thèse de F. Varret \cite{varret-these}}.

Le calcul des spectres théoriques se fait en 4 étapes :

\begin{itemize}
\item \'Ecriture des éléments de matrice des hamiltoniens de l'état excité et de l'état fondamental,
\item Recherche des énergies (valeurs propres) et des fonctions d'ondes (vecteurs propres) associées à ces hamiltoniens,
\item Calcul des énergies et des intensités des raies de transition
\item Habillage des raies.
\end{itemize}

\subsubsection{\'Eléments de matrice du Hamiltonien}
\begin{multicols}{2}
Le Hamiltonien est représenté par une matrice hermitique.
Les matrices étant symétriques, on ne représente que les éléments situées au dessus de la diagonale, 
selon la numérotation figurée ci-contre :

\columnbreak

\begin{tabular}{|c|c|c|c|c|}
  \hline
    1 & 2 & 4 & 7 & 11\\
  \hline
      & 3 & 5 & 8 & \dots\\
  \hline
      &   & 6 & 9 & \\
  \hline
      &   &   & 10 &\\
  \hline
\end{tabular}
\end{multicols}

Les axes choisis sont les axes principaux du gradient de champ électrique. 
Le champ interne (\lstinline{CH}) est noté $H_n$, dont les angles polaires (\lstinline{THETA} et \lstinline{GAMA})  sont notés $\theta,\gamma$ (Fig \ref{fig:theta_gamma}).
Le paramètre d'asymétrie (\lstinline{ETA}) est noté $\eta$, l'interaction quadrupolaire (\lstinline{SQ}) est $(e^2 q_z Q/2)\sqrt{1+\eta^2 /3}$.



\begin{figure}[!h]
\centering
\includegraphics[scale=0.4]{triedre_thga.png}
\caption{\label{fig:theta_gamma} Champ interne dans le repère $OXYZ$}
\end{figure}

Les composantes du champ interne sont :
\begin{align*}
  H_x &= H_n \sin \theta \cos \gamma,\\
  H_y &= H_n \sin \theta \sin \gamma,\\
  H_z &= H_n \cos\theta.
\end{align*}

\paragraph{\'Etat fondamental}(I=1/2)

\begin{tabular}{ccrcl}
$\mathcal{H}_f(1)$&$=$&    $\left\langle 1/2 \big\vert 1/2 \right\rangle$ &$=$& $\displaystyle-(1/2)\hbar \gamma_f H_z$, \\
$\mathcal{H}_f(2)$  &$=$&$\left\langle 1/2 \big\vert -1/2 \right\rangle$ &$=$&$\displaystyle -(1/2)\hbar \gamma_f \left(H_x -i H_y\right)$, \\
$\mathcal{H}_f(3)$  &$=$&$\left\langle -1/2 \big\vert -1/2 \right\rangle$ &$=$& $\displaystyle(1/2)\hbar \gamma_f H_z.$ \\
\end{tabular}

\paragraph{\'Etat excité}(I=3/2)

 On note $C=(e^2 q_z Q/4) = $\lstinline{SQ}$/(2\sqrt{1+\eta^2/3})$
  
\begin{tabular}{ccrcl} 
$\mathcal{H}_e(1)$&$=$&$ \left\langle 3/2 \big\vert 3/2 \right\rangle$ &$=$& $\displaystyle-(3/2)\hbar \gamma_e H_z +C$,\\
$\mathcal{H}_e(1)$&$=$&$ \left\langle 3/2 \big\vert 1/2 \right\rangle$ &$=$& $\displaystyle-(\sqrt{3}/2)\hbar \gamma_e \left(H_x - i H_y\right)$,\\
$\mathcal{H}_e(1)$&$=$&$ \left\langle 1/2 \big\vert 1/2 \right\rangle$ &$=$& $\displaystyle-(1/2)\hbar \gamma_e H_z -C$,\\
$\mathcal{H}_e(1)$&$=$&$ \left\langle 3/2 \big\vert -1/2 \right\rangle$ &$=$& $\displaystyle(1/\sqrt{3})\eta C$,\\
$\mathcal{H}_e(1)$&$=$&$ \left\langle 1/2 \big\vert -1/2 \right\rangle$ &$=$& $\displaystyle-(1/2)\left( H_x - i H_y  \right)$,\\
$\mathcal{H}_e(1)$&$=$&$ \left\langle -1/2 \big\vert -1/2 \right\rangle$ &$=$& $\displaystyle-(1/2)\gamma_e H_z -C$,\\
$\mathcal{H}_e(1)$&$=$&$ \left\langle 3/2 \big\vert -3/2 \right\rangle$ &$=$& $\displaystyle0$,\\
$\mathcal{H}_e(1)$&$=$&$ \left\langle 1/2 \big\vert -3/2 \right\rangle$ &$=$& $\displaystyle(1/\sqrt{3})\eta C$,\\
$\mathcal{H}_e(1)$&$=$&$ \left\langle -1/2 \big\vert -3/2 \right\rangle$ &$=$& $\displaystyle-(\sqrt{3}/2)\hbar \gamma_e \left(H_x - i H_y\right)$, \\
$\mathcal{H}_e(1)$&$=$&$ \left\langle -3/2 \big\vert -3/2 \right\rangle$ &$=$& $\displaystyle-(3/2)\hbar \gamma_e H_z + C$, \\
\end{tabular}

\subsubsection{Fonctions d'onde et énergies}
{\it Partie IV.1 de la thèse de F.Varret}

La diagonalisation des matrices $\mathcal{H}_f$ et $\mathcal{H}_e$ donne les énergies et fonctions d'onde des niveaux nucléaires fondamentaux $\left| F\right\rangle$ et excités $\left| E\right\rangle$ dans les axes $OXYZ$.
On obtient alors par différence les 8 énergies de transition possibles entre les 2 niveaux nucléaires fondamentaux ($I=1/2$) et les 4 niveaux nucléaires excités (I=3/2) ; nous avons donc la position des 8 raies Mössbauer, dont certaines peuvent éventuellement êtres confondues, ou d'intensité nulle. 

Les intensités de raies se calculent de manière différente dans le cas du monocristal et dans le cas de la poudre. 

\paragraph{Cas du monocristal}
Lorsque l'absorbeur est monocristallin, les intensités des raies dépendent de l'orientation du rayonnement dans le trièdre $OXYZ$ lié à l'absorbeur (Fig \ref{fig:alpha_beta}).
Il est avantageux de considérer un autre trièdre de référence $OX'Y'Z'$ où OZ' est parallèle à la direction de propagation du rayonnement.

\begin{figure}[!h]
\centering
\includegraphics[scale=0.4]{triedre_albe.png}
\caption{\label{fig:alpha_beta}Direction de propagation $OZ'$ dans le triedre $OXYZ$}
\end{figure}

Dans cette nouvelle référence, les fonctions d'onde nucléaires deviennent $\left| F'\right\rangle$ et $\left| E'\right\rangle$, et l'on peut montrer que dans le cas du rayonnement qui est le notre (photon de $L=1$), l'intensité correspondant à la transition considérée est donnée par:
$$ I(niv.f \rightarrow niv.e) \propto \left| \left\langle F'\right| M^{+1} \left|E'\right\rangle \right|^2  + \left| \left\langle F'\right| M^{-1} \left|E'\right\rangle \right|^2 $$

Les éléments de matrice de $ M^{+1}$ et $ M^{-1}$, tabulés ci-dessous, sont des coefficients de Glebsch-Gordan
 du type $\left\langle I_f m_f LM/I_e m_e \right\rangle$
 intervenant dans la probabilité de transition d'un niveau  $\left\langle I_f m_f\right|$ vers un niveau  $\left\langle LM/I_e m_e \right|$ 
 sous l'effet d'un photon dont le moment orbital $L$ a pour projection $M$ sur la direction de propagation (des expressions  plus générales des probabilités de transistion sont données dans le livre d'Abragam \cite{abragam-appli}, p34)

Il faut donc opérer un changement d'axes sur les fonctions d'onde nucléaires  ($\left|F\right\rangle$ et $\left|E\right\rangle$ dans les axes $OXYZ$ du gradient,  $\left| F'\right\rangle$ et $\left| E'\right\rangle$ dans les axes $OX'Y'Z'$ du rayonnement).
Ceci s'obtient à l'aide des matrices de rotation $\mathcal{R}(1/2)$ et $\mathcal{R}(3/2)$; les angles d'Euler de la rotation amenant $OXYZ$ sur $OX'Y'Z'$ sont alors $\alpha, \beta, 0$ ; et on a :


$$ I(f \rightarrow e) \propto \left| \left\langle F\right|\mathcal{R}(1/2)\, M^{+1}\,\mathcal{R}^{\dagger}(3/2) \left|E\right\rangle \right|^2 
 + \left| \left\langle F\right| \mathcal{R}(1/2)\, M^{-1}\,\mathcal{R}^{\dagger}(3/2) \left|E\right\rangle \right|^2 $$

soit 
\begin{equation}
\label{eq:monoc}
I(f \rightarrow e) \propto \left| \left\langle F\right| \mathcal{M}^{+1} \left|E\right\rangle \right|^2  
      + \left| \left\langle F\right| \mathcal{M}^{-1} \left|E\right\rangle \right|^2
\end{equation}
où  $\mathcal{M}^{+1}$ et $\mathcal{M}^{-1}$ sont les matrices transmuées de $M^{+1}$ et $M^{-1}$ dans le changement d'axe. 
Les éléments de matrice de $\mathcal{M}^{+1}$ et $\mathcal{M}^{-1}$ sont :

\begin{equation*}
\mathcal{M}^{+1}=\left(\begin{array}{cccc}
e^{i\alpha}(1-\cos{\beta})/2~,~~ -\sin{\beta}/\sqrt{3} ~,~~ e^{-i\alpha}( 1+\cos{\beta})/(2\sqrt{3})~,~~0\\
\\
0~,~~ e^{i\alpha}(1-\cos{\beta})/(2\sqrt{3})~,~~-\sin{\beta}/\sqrt{3}~,~~ e^{-i}\alpha(1-\cos{\beta})/2
\end{array}\right)
\end{equation*}

\begin{equation*}
\mathcal{M}^{-1}=\left(\begin{array}{c}
e^{i\alpha}(1+\cos{\beta})/2~,~~ \sin{\beta}/\sqrt{3}~,~~e^{-i\alpha}( 1-\cos{\beta})/(2\sqrt{3})~,~~0\\
\\
0~,~~e^{i\alpha}(1+\cos{\beta})/(2\sqrt{3})~,~~\sin{\beta}/\sqrt{3}~,~~ e^{-i\alpha}(1-\cos{\beta})/2
\end{array}\right)
\end{equation*}
Utilisant l'équation \eqref{eq:monoc} il est donc possible de calculer algébriquement la dépendance angulaire de l'intensité de chaque raie vis-à-vis de l'orientation du rayonnement. 

\paragraph{Cas de la poudre} (Répartition isotrope de l'orientation des cristallites)

Lorsque l'absorbeur est en poudre, on peut montrer (\cite{gabriel-methodo})  que l'intensité d'une raie correspond à la formule 

\begin{equation}
I(f \rightarrow e) \propto \left| \left\langle F'\right| M^{+1} \left|E'\right\rangle \right|^2  
      + \left| \left\langle F'\right| M^{0} \left|E'\right\rangle \right|^2
      + \left| \left\langle F'\right| M^{-1} \left|E'\right\rangle \right|^2
\end{equation}

Où $M^0$ a pour éléments de matrice : 
\begin{equation}
M^0=\left(\begin{array}{ccccccc}
  0&,&\sqrt{2}/\sqrt{3} &,&0&,& 0 \\
  \\
  0&,&0&,&\sqrt{2}/\sqrt{3} &,& 0
\end{array} \right)
\end{equation}

En outre la somme apparaissant dans l'équation 4 est indépendante de l'orientation des axes $OX'Y'Z'$ (c'est à dire de la direction du rayonnement, ce qui est naturel).
Aussi tout changement d'axes est-il inutile, et on peut utiliser :

\begin{equation}
I(f \rightarrow e) \propto \left| \left\langle F\right| M^{+1} \left|E\right\rangle \right|^2  
      + \left| \left\langle F\right| M^{0}  \left|E\right\rangle \right|^2
      + \left| \left\langle F\right| M^{-1} \left|E\right\rangle \right|^2
\end{equation}.
\subsubsection{Habillage des raies}
Il ne reste plus qu'à introduire le profil des raies; la théorie de l'absorption (ou émission) résonnante donne aux raies du spectre un profil Lorentzien et une largeur à mi-hauteur de $\approx0.2~\milli\meter\reciprocal\second$ environ (\cite{abragam-appli} p.37) déduite de la durée de vie de l'état nucléaire excité.

Dans la pratique on n'observe jamais cette largeur "naturelle", par suite d'élargissement d'origine microscopique notament, comme une legère dispersion des valeurs hyperfines,
 ou d'origine macroscopique , comme les effets de saturation dans les absorbeurs épais, ou même d'origine instrumentale.
 
 On procède donc à l'"habillage" du spectre par des raies lorentziennes dont la largeur n'est pas fixée a priori. On peut également choisir d'habiller les raies par une convolution gaussienne*lorentzienne (le choix se fait via l'option \lstinline{IO(16)}).

\subsection{Cyclo\"ide}
Mosfit2016 permet le calcul de spectres dans le cas d'un ordre magnétique spiral dans l'axe $c$ du cristal, de pas incommensurable. 
Le vecteur de propagation est dans le plan de rotation des moments.
Le calcul se fait dans les axes principaux du gradient de champ électrique.
Le champ hyperfin possède une composante de norme $H_\parallel$ dans la direction de l'axe $OZ$, et une composante de norme $H_\perp$ dans le plan $OXY$.
Le champ $H_n$ est la projection avec anisotropie du champ hyperfin dans le plan $XOZ$ avec une distribution uniforme de l'angle $\theta$.
On admet une dépendance anisotrope :
\begin{equation}
  H_n = H_{\parallel}\cos^2  \theta + H_{\perp}\sin^2\theta
\end{equation}
On pose le paramètre cyclo\"idal $\omega_m = H_{\perp}/ H_{\parallel}$.
$H_n$ s'écrit alors :
\begin{equation}
  H_n = H_{\parallel}( \cos^2\theta + \omega_m \sin^2 \theta)
\end{equation}
Les composantes de $H_n$ sur les axes $OXYZ$ sont alors :
\begin{align*}
  H_x &= H_n \sin \theta,\\
  H_z &= H_n \cos \theta,\\
  H_y &= 0.
\end{align*}

\begin{figure}[!h]
\centering
\includegraphics[scale=0.4]{triedre_cyclo.png}
\caption{\label{fig:cycloide}$H_n$ dans le plan $OXZ$}
\end{figure}

Pour calculer un spectre, on va appliquer cette formulation du champ $H_n$ pour une distribution de valeur de $\theta$ de $0$ à $2\pi$. 
Le spectre total est la moyenne de tous les spectres obtenus.
On doit alors fournir au programme une valeur pour $H_{\parallel}$ (qui sera contenue dans la variable \lstinline{CH}) et une valeur pour le paramètre cyclo\"idal $\omega_m$ (qui sera contenue dans \lstinline{GAMA}).
Les variables \lstinline{ALPHA} et \lstinline{BETA} doivent etre fixée à zéro.

