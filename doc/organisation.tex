\FloatBarrier
\subsection{Modules}
Le programme est écrit selon la normae Fortran2003. 
Les sous-soutines sont reparties dans des modules en fonction de leur role
et des variables qu'elles font intervenir (Voir table \ref{tab:role_module}).

Les variables qui étaient partagés entre les sous-routines à l'aide de l'instruction \lstinline{COMMON} dans la version Fortran77 de Mosfit
sont à présent déclarées comme variables globales certains modules.
Une variables globale d'un module \lstinline{exemple} est accessible depuis les sous-routines de tout les modules appelant le module \lstinline{exemple} (instruction \lstinline{use exemple}).

\begin{table}
\begin{tabular}{|r|l|}
\hline
Nom du module & R\^ole du module\\ \hline \hline
 \lstinline{ajustement} & Algorithme de moindres carrés, vérification de la convergence \\\hline
 \lstinline{algebre} & Méthodes d'algèbre linéaire\\\hline
 \lstinline{connex}	& Connexions entre paramètres \\\hline
 \lstinline{ecriture} &Routines d'écritures dans les fichiers\\\hline
 \lstinline{habillage} &Habillage des raies par des lorentziennes\\
          & ou des convolutions lorentzienne*gaussiennes\\\hline
  \lstinline{hamiltonien} & Calcul des hamiltonien, des fonctions d'ondes,\\
          & des énergies et des intensités \\\hline
 \lstinline{lecture} &Lecture du fichier d'entrée\\\hline
 \lstinline{main} & Programme principal \\ \hline
 \lstinline{options} & Déclaration des variables de spécification des options\\\hline
 \lstinline{spectres}	& Manipulation des spectres : création des spectres théoriques, \\
        &  manipulation des sous-spectres, calculs de dérivée des spectres\\\hline
 \lstinline{variablesAjustables} & Ensemble des paramètres hyperfins \\
      & et des variables pouvant être affinées par la méthode des \\
      &moindres carrés, tableaux de rangement de ces variables\\\hline
 \lstinline{variablesFixes}	& Données fixes du problème \\ \hline
\end{tabular}
\caption{\label{tab:role_module}Nom et rôle des modules}
\end{table}

\subsection{nomenclature}
\paragraph{Modules :}
Les noms de modules sont écrits avec la casse lowerCamelCase (mots reunis sans espace ni {\it underscore}, avec majuscule pour identifier le début des mots) : \lstinline{nomDuModule}.
Le nom du module doit indiquer le r\^ole général des routines et des variables qu'il contient.

\paragraph{Sous-routines :} 
Les sous-routines sont nommées en fonction du module auquel elles appartiennent.
Le nom d'une sous-routine nom doit indiquer de manière claire et succinte l'opération principale qu'elle effectue.
Ainsi, une sous-routine appartenant au module \lstinline{moduleExemple} qui remplie la fonction d'afficher \lstinline{N} messages aléatoires pourra s'appeler 

\lstinline{moduleExemple_afficher_messages_aleatoires}

 et être appelée par
 
 \lstinline{ call  moduleExemple_afficher_messages_aleatoires(N)}

Les fonctions étant destinées à être appelées au cours d'aures instructions, on prèfère choisir un nom plus court.

\paragraph{Variables :}
Pour une meilleure comprehension , les noms des variables globales des modules sont écrites en majuscules, tandis que les noms des variables locales des sous-routines sont écrites en minuscule.
Fortran est indifférent à la casse, ce choix majuscules/minuscules est donc purement esthétique, afin de faciliter la comprehension du developpeur.

Exemple : 
\begin{itemize}
\item Dans le module \lstinline{spectres}, \lstinline{CN} est une variable globale du module \lstinline{variablesFixes} rendue accessible par l'instruction \lstinline{use variablesFixes}.
\item Dans le module \lstinline{lecture}, le module \lstinline{variablesFixes} n'est pas utilisé.
 La variable \lstinline{cn} est alors une variable locale de la sous-routine \lstinline{lecture_options}  dont la valeur est transmise en tant qu'argument. 
\end{itemize}
La variable \lstinline{dp}, qui indique le \lstinline{kind} de la double precision, est l'exception à cette règle de casse.

\subsection{Documentation interne du code}

Utilisation de Doxygen ?
description du role des modules
-routines de lecture/ecriture


\FloatBarrier
